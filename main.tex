\pdfminorversion = 4
\documentclass[shortpres,aspectratio=43]{beamer}
%\documentclass[shortpres,aspectratio=169]{beamer}
\usetheme{CambridgeUS}


\setbeamertemplate{footline}
{
  \leavevmode%
  \hbox{%
  \begin{beamercolorbox}[wd=.333333\paperwidth,ht=2.25ex,dp=1ex,left]{author in head/foot}%
  \hspace*{4ex}\usebeamerfont{author in head/foot}Niklas Kochdumper%~~\beamer@ifempty{\insertshortinstitute}{}{(\insertshortinstitute)}
  \end{beamercolorbox}%
  \begin{beamercolorbox}[wd=.333333\paperwidth,ht=2.25ex,dp=1ex,center]{title in head/foot}%
    \usebeamerfont{title in head/foot} \insertshortdate{}
  \end{beamercolorbox}%
  \begin{beamercolorbox}[wd=.333333\paperwidth,ht=2.25ex,dp=1ex,right]{date in head/foot}%
    \usebeamerfont{date in head/foot}
    \insertframenumber{} / \inserttotalframenumber\hspace*{2ex}
  \end{beamercolorbox}}%
  \vskip0pt%
}\part{title}
\beamertemplatenavigationsymbolsempty

%color specification-----------------------------------------------------
\definecolor{TUMblue}{RGB}{27, 94, 170}%{rgb}{0.00, 0.40, 0.74}
\definecolor{TUMgray}{rgb}{0.85, 0.85, 0.86}
\definecolor{TUMpantone285C}{rgb}{0.00, 0.45, 0.81}
\definecolor{TUMpantone300C}{RGB}{27, 94, 170} %uncorrected TUMpantone300C
\definecolor{lightblue}{RGB}{213,227,241}%{rgb}{0.7529,0.8118,0.9333}
\definecolor{darkgreen}{RGB}{0, 200, 0}
\definecolor{verydarkgreen}{RGB}{0 120 0}
\definecolor{SBred}{RGB}{150 0 0}

\setbeamercolor{block title}{fg=white, bg=SBred}
\setbeamercolor{block body}{bg=lightblue}
\setbeamertemplate{blocks}[rounded][shadow=true]

%------------------------------------------------------------------------
\setbeamercolor{frametitle}{bg=SBred, fg=white}
\setbeamercolor{palette primary}{bg=SBred, fg=white}%{fg=TUMblue,bg=TUMgray}
\setbeamercolor{palette secondary}{use=palette primary,bg=SBred, fg=white}
\setbeamercolor{palette tertiary}{use=palette primary,fg=white, bg=SBred}
\setbeamercolor{palette quaternary}{use=palette primary,fg=white, bg=SBred}

\setbeamercolor{title}{bg=SBred,fg=white}
\setbeamercolor{item projected}{use=item,fg=black,bg = lightblue}
\setbeamercolor{block title}{fg=white, bg=SBred}
\setbeamercolor{block body}{bg=white}
\setbeamertemplate{blocks}[rounded][shadow=true]

%------------------------------------------------------------------------
\setbeamertemplate{bibliography item}{\insertbiblabel}
\setbeamercolor{bibliography item}{parent=palette primary}
\setbeamercolor{bibliography entry author}{fg=TUMblue}

%------------------------------------------------------------------------
\setbeamertemplate{enumerate items}[bullet]

%------------------------------------------------------------------------
\usepackage{subfigure}
\usepackage{textpos} % for figure (logo) on slides
\usepackage{psfrag} % for \psfrag in figures
\usepackage{algorithm,algpseudocode} % for algorithm environment
\usepackage{booktabs} % for rulers in tables
\usepackage{units} % for units to values
%\usepackage{hyperref}
\usepackage{tikz}
\usetikzlibrary{calc,arrows,arrows.meta, automata,positioning,backgrounds}
\usetikzlibrary{overlay-beamer-styles,patterns}
\usetikzlibrary{chains, trees, fit, tikzmark}
\usetikzlibrary{shapes.geometric, shapes.arrows, decorations.pathreplacing}
\usepackage{makecell}
\usepackage{pgfplots}
\usepackage{graphicx}
\usepackage{colortbl}
\usepackage{array}
\usepackage{booktabs}
\usepackage{multirow}
\usepackage{bbding}
\usepackage{german}
\usepackage[export]{adjustbox}
\usepackage{pifont}

\newcommand{\cmark}{\ding{51}}%
\newcommand{\xmark}{\ding{55}}%

% TikZ ---------------
\tikzset{
	preEdge/.style={black, <-, shorten <=1pt, >=stealth', semithick, rounded corners=2pt},
	postEdge/.style={black, ->, shorten >=1pt, >=stealth', semithick, rounded corners=2pt},
	axes/.style={black, ->, >=stealth'}
}
\definecolor{setedgeblue}{rgb}{0.00, 0.45, 0.81}
\definecolor{setfaceblue}{rgb}{0.7529,0.8118,0.9333}
\definecolor{setedgegreen}{rgb}{0.00,0.50,0.00}
\definecolor{setfacegreen}{rgb}{0.4510,0.9020,0.4510}
\definecolor{setedgegray}{rgb}{0.50,0.50,0.50}
\definecolor{setfacegray}{rgb}{0.85,0.85,0.85}
\definecolor{setedgered}{rgb}{0.6992,0.1367,0.1367}
\definecolor{setfacered}{rgb}{0.8906,0.2617,0.2070}
% --------------------

\newcommand{\arstretch}[1]{\renewcommand{\arraystretch}{#1}}
\newcolumntype{C}[1]{>{\centering\arraybackslash}m{#1}}

%-----------------------------------------------------------------------
\newcommand{\at}{\fontfamily{ptm}\selectfont @}
\newcommand{\ra}[1]{\renewcommand{\arraystretch}{#1}} %to change the row spacing in tables

\newcommand\blfootnote[1]{%
  \begingroup
  \renewcommand\thefootnote{}\footnote{#1}%
  \addtocounter{footnote}{-1}%
  \endgroup
}

\setbeamertemplate{headline}[default]

%-----------------------------------------------------------------------
\title[AutoKoopman]{AutoKoopman: A Toolbox for Automated System Identification via Koopman Operator Linearization}

% \author[Lew et al.]{Stanley Bak$^1$, Sergiy Bogomolov$^2$, Brandon Hencey$^3$, \\ Niklas Kochdumper$^1$, \underline{Ethan Lew}$^4$, and Kostiantyn Potomkin$^2$}

\author[Lew et al.]{\underline{Ethan Lew}$^1$, Abdelrahman Hekal$^2$, Kostiantyn Potomkin$^2$, Niklas Kochdumper$^3$, Brandon Hencey$^4$, Stanley Bak$^3$, Sergiy Bogomolov$^2$}

\institute[Galois Inc.]{$^1$Galois Inc., $^2$Newcastle University, \\$^3$ Stony Brook University, $^4$Air Force Research Laboratory}

\date[ATVA 2023]{}

%---------------------------------------------------------------------
\begin{document}

%% TUM logo
%\addtobeamertemplate{frametitle}{}{
%\begin{textblock*}{\textwidth}(.91\textwidth,-0.8cm) % for aspectratio=43
%\includegraphics[height=0.65cm]{./figures/logo.png} % for aspectratio=43
%\begin{textblock*}{\textwidth}(.95\textwidth,-0.9cm) % for aspectratio=169
%\includegraphics[height=0.8cm]{./figures/logo.png} % for aspectratio=169
%\end{textblock*}}


\begin{frame}[plain]
    \titlepage
\end{frame}

% Slide 2: Motivations
\begin{frame}
\frametitle{Motivations}
\begin{itemize}
    \item<1-> Koopman linearization provides a valuable method for identifying continuous dynamical systems using observed data.
    \item<2-> It captures complex non-linear behaviors, providing a globally applicable linear model.
    \item<3-> Useful for future state prediction, system design and optimization, and enhancing interpretability and physical understanding.
    \item<4-> \textbf{Challenge:} System identification via Koopman linearization is complex, requiring careful selection of observables and hyperparameters.
    \item<5-> \textbf{Solution:} AutoKoopman automates this, optimizing all parameters to deliver an accurate Koopman linearized model.
\end{itemize}
\end{frame}

% Slide 3: System Identification Problem
\begin{frame}
\frametitle{System Identification Problem}
We aim to identify the dynamics of a continuous-time system from a set of trajectories:
\begin{equation}
\dot{\mathbf{x}}(t) = \mathbf{f}(\mathbf{x}(t), \mathbf{u}(t))
\end{equation}
\pause
Given trajectories, our goal is to learn a function \textbf{f} predicting the next state or time derivative. The system dynamics can be described by an arbitrary black-box function $\mathbf{f}(\mathbf{x}(t), \mathbf{u}(t))$, accommodating various system types.
\end{frame}

% Slide 4: The Koopman Operator
\begin{frame}
\frametitle{The Koopman Operator}
In 1931, B.O. Koopman introduced an alternative way to represent dynamic systems. The Koopman operator offers a globally linear representation, but only for specific systems.
\pause
\[
\mathcal{K}_{t} \big(\mathbf{g} (\mathbf{x}_0, \mathbf{u}_0)\big) = \mathbf{g} \big(\mathbf{F}_{t}(\mathbf{x}_0,  \mathbf{u}_0 ), \mathbf{u}(t) \big)
\]
\pause
We utilize explicit learning techniques for the operator. Our aim is a finite space $\mathcal{H}$ where system dynamics are well approximated by a linear operator.
\pause
The optimization problem seeks the observables $\mathbf{g}$ and the Koopman operator $K_{\Delta t}$ minimizing the mean square error over the trajectories set $\mathcal{T}$.
\end{frame}


\end{document}